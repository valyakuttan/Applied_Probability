\documentclass{tufte-handout}
\usepackage{amsmath, amsfonts, amsthm, amssymb}
\usepackage{thmtools}


%% environments
\theoremstyle{definition} \newtheorem{definition}{Definition}
\newtheorem{theorem}{Theorem}
\theoremstyle{definition} \newtheorem{remark}{Remark}

% theorem style for example
\declaretheoremstyle[
spaceabove=6pt, spacebelow=6pt,
headfont=\normalfont\bfseries,
notefont=\mdseries, notebraces={(}{)},
bodyfont=\normalfont,
postheadspace=1em,
%numberwithin=section
]{exstyle}

% theorem style for solution
\declaretheoremstyle[
spaceabove=6pt, spacebelow=6pt,
headfont=\normalfont\bfseries,
notefont=\mdseries, notebraces={(}{)},
bodyfont=\normalfont,
postheadspace=1em,
headpunct={},
qed= $\blacksquare$,
numbered=no
]{solstyle}

\declaretheorem[style=exstyle]{example}
\declaretheorem[style=solstyle]{solution}


%% commands
\newcommand{\prob}[1]{\mathbf{P} \left( #1 \right)}
\newcommand{\cprob}[2]{\mathbf{P} \left( #1 \, | \, #2 \right)}
\newcommand{\pmf}[2]{p_{#1} \left( #2 \right)}
\newcommand{\expt}[1]{\mathbf{E} \left[ #1 \right]}
\newcommand{\jpmf}[4]{p_{#1, #2} \left( #3, #4 \right)}
\newcommand{\cpmf}[3]{p_{#1 | #2} \left( #3 \right)}
\newcommand{\cexpt}[2]{\mathbf{E} \left[ #1 \, | \, #2 \right]}

\author{John N. Tsitsiklis}
\title{Joint PMF and Conditioning of Multiple Random Variables}
\begin{document}
\maketitle
\section{Joint PMF of Multiple Random Variables}
\begin{definition}[Joint PMF]
  Let $X$ and $Y$ be two random variables associated with the same
  experiment. Then the \emph{joint PMF} $p_{X, y}$ of $X$ and $Y$ is
  defined by
  \begin{equation*}
    \jpmf{X}{Y}{x}{y} = \prob{X = x, Y = y}
  \end{equation*}
\end{definition}

\begin{definition}[Marginal PMF]
  The \emph{marginal PMF}s of $X$ and $Y$ can be obtained from the joint
  PMF, using the formulas
  \begin{equation*}
    \pmf{X}{x} = \sum_y \jpmf{X}{Y}{x}{y}, \quad
    \pmf{Y}{y} = \sum_x \jpmf{X}{Y}{x}{y}
  \end{equation*}

\end{definition}

\begin{remark}
  If $A$ is the set of all pairs that have a certain property, then
  \begin{equation*}
      \prob{A} = \sum_{(x, y) \in A} \jpmf{X}{Y}{x}{y}
  \end{equation*}
\end{remark}

\begin{remark}
  A function $g(X, Y)$ of $X$ and $Y$ defines another random variable, and
  \begin{equation*}
    \expt{g(X, Y)} = \sum_x \sum_y g(x, y) \jpmf{X}{Y}{x}{y}
  \end{equation*}
\end{remark}

\begin{remark}
  If $g(X, Y) = aX + bY + c$, we have
  \begin{equation*}
    \expt{aX + bY + c} = a \expt{X} + b \expt{Y} + c
  \end{equation*}
\end{remark}

\begin{example}[The Hat Problem]
  Suppose that $n$  people throw their hats in a box and then each picks
  one hat at random. What is the expected value of $X$, the number of
  people that get back their own hat?
\end{example}

\begin{solution}
  For the $i$th person, let's define the random varible $X_i$ such that
  \begin{equation*}
    X_i =
    \begin{cases}
      1 & \text{if the person picks his own hat},\\
      0 & \text{otherwise}.
    \end{cases}
  \end{equation*}

  Since $\prob{X_i = 1} = \frac{1}{n}$, we have
  \begin{equation*}
    \expt{X_i} = 1 \cdot \frac{1}{n} + 0 \cdot (1 - \frac{1}{n})
    = \frac{1}{n}
  \end{equation*}

  We now have
  \begin{equation*}
    X = X_1 + X_2 + \cdots + X_n
  \end{equation*}

  so that
  \begin{equation*}
    \expt{X} = \expt{X_1} + \expt{X_2} + \cdots + \expt{X_n}
    = n \cdot \frac{1}{n} = 1
  \end{equation*}
\end{solution}

\section{Conditioning}

\subsection{Conditioning a Random Varible on an Event}

\begin{definition}[Conditional PMF]
  The \emph{Conditional PMF} of a random variable $X$, conditioned on a
  particular event $A$ with $\prob{A} > 0$, is defined by
  \begin{equation*}
    \cpmf{X}{A}{x} = \cprob{X = x}{A}
    = \frac{ \prob{\{X = x \} \cap A}}{\prob{A}}
  \end{equation*}
\end{definition}

\begin{remark}
  The events $\{X = x\} \cap A$ are disjoint for different values of $x$,
  their union is $A$, hence
  \begin{equation*}
    \prob{A} = \sum_x \prob{\{X = x\} \cap A}
  \end{equation*}

  Combining the above formulas we have
  \begin{equation*}
    \sum_x \cprob{X}{A}{x} = 1
  \end{equation*}

  So $p_{X|A}$ is a legitimate PMF.
\end{remark}

\begin{remark}
  If $A_1, \ldots, A_n$ are disjoint events that form a partition of the
  sample space, with $\prob{A_i} > 0$ for all $i$, then
  \begin{equation*}
    \pmf{X}{x} = \sum_{i = 1} ^n \prob{A_i} \cpmf{X}{A_i}{x}
  \end{equation*}

  Also for any event $B$, with $\prob{A_i \cap B} > 0$ for all $i$, we
  have
  \begin{equation*}
    \cpmf{X}{B}{x} = \sum_{i = 1} ^n \cprob{A_i}{B} \cpmf{X}{A_i \cap B}{x}
  \end{equation*}
\end{remark}

\subsection{Conditioning one Random Variable on Another}

\begin{definition}
  Let $X$ and $Y$ be two random variables associated with the same
  experiment. The \emph{Conditional PMF} $p_{X|Y}$ of $X$ given $Y$ is
  defined by
  \begin{equation*}
    \cpmf{X}{Y}{x|y} = \prob{X = x \, | \, Y = y}
  \end{equation*}
\end{definition}

\begin{remark}
  \begin{equation*}
    \jpmf{X}{Y}{x}{y} = \pmf{Y}{y} \cpmf{X}{Y}{x|y}
    = \pmf{X}{x} \cpmf{Y}{x}{y|x}
  \end{equation*}
\end{remark}

\subsection{Conditional Expectation}

Let $X$ and $Y$ be random variables associated with the same experiment
\begin{itemize}
\item The \emph{conditional expectation} of $X$ given an event $A$ with
  $\prob{A} > 0$, is defined by
  \begin{equation*}
    \cexpt{X}{A} = \sum_x x \cpmf{X}{A}{x}
  \end{equation*}

  For a function $g(X)$, we have
    \begin{equation*}
    \cexpt{g(X)}{A} = \sum_x g(x) \cpmf{X}{A}{x}
  \end{equation*}

\item The conditional expectation of $X$ given a value $y$ of $Y$ is
  defined by
  \begin{equation*}
    \cexpt{X}{Y = y} = \sum_x x \cpmf{X}{Y}{x | y}
  \end{equation*}

\item   If $A_1, \ldots, A_n$ are disjoint events that form a partition
  of the sample space, with $\prob{A_i} > 0$ for all $i$, then
  \begin{equation*}
    \expt{X} = \sum_{i = 1} ^n \prob{A_i} \cexpt{X}{A_i}
  \end{equation*}

  Also for any event $B$, with $\prob{A_i \cap B} > 0$ for all $i$, we
  have
  \begin{equation*}
    \cexpt{X}{B} = \sum_{i = 1} ^n \cprob{A_i}{B} \cexpt{X}{A_i \cap B}
  \end{equation*}

\item We have
  \begin{equation*}
    \expt{X} = \sum_y \pmf{Y}{y} \cexpt{X}{Y = y}
  \end{equation*}
\end{itemize}

\begin{example}[D. Bernoulli's problem of joint lives]
  Consider $2m$ persons forming $m$ couples who live together at a given
  time. Suppose that at some later time the probability of each person
  being alive is $p$, independent of other persons. At that later time,
  let $A$ be the number of persons that are alive and let $S$ be the
  number of couples in which both partners are alive. For any survivor
  number $a$, find $\cexpt{S}{A = a}$
\end{example}

\begin{solution}
  Let $X_j$ be the indicator of the event that the
  $j$th couple survives. Then
  \begin{align*}
    \cexpt{S}{A = a} & = \expt{\sum_{j = 1} ^m X_j} \\
                     & = m \expt{X_1}\\
                     & = m \prob{X_1 = 1}
  \end{align*}
  because the chance of survival is the same for every couple. Now we can
  choose the $a$ survivors in $\binom{2m}{a}$ ways, and the number of
  these in which the first couple is remain alive is
  $\binom{2m - 2}{a - 2}$. Since all outcomes are equally likely we have
  \begin{equation*}
    \prob{X_1 = 1} = \frac{\binom{2m - 2}{a - 2}}{\binom{2m}{a}}
    = \frac{a(a - 1)}{2m(2m - 1)}
  \end{equation*}
  Hence we have
  \begin{equation*}
    \cexpt{S}{A = a} = m \cdot \frac{a(a - 1)}{2m(2m - 1)}
    = \frac{a(a - 1)}{2(2m - 1)} 
  \end{equation*}

  \noindent \textbf{An alternative solution}:\\
  Let $X_i$ be the random variable taking the value
  $1$ or $0$ depending on whether the first partner of the $i$th couple
  has survived or not. Let $Y_i$ be the corresponding random variable for
  the second partner of the $i$th couple. Then, we have
  $S = \sum_{i=1} ^m X_i Y_i$ , and by using the total expectation theorem,
  \begin{align*}
    \cexpt{S}{A = a} & = \sum_{i = 1} ^m \cexpt{X_i Y_i}{A = a}\\
                     & = m \cexpt{X_1 Y_1}{A = a}\\
                     & = m \cprob{X_1Y_1 = 1}{A = a}
  \end{align*}
  Also
  \begin{align*}
    \cprob{X_1Y_1 = 1}{A = a} & = \cprob{Y_1 = 1}{X_1 = 1 \cap A = a}
                                \cdot \cprob{X_1 = 1}{A = a}\\
  \end{align*}
  We have
  \begin{equation*}
    \cprob{Y_1 = 1}{X_1 = 1 \cap A = a} = \frac{a - 1}{2m - 1}, \quad
    \cprob{X_1 = 1}{A = a} = \frac{a}{2m}
  \end{equation*}
  Thus
  \begin{align*}
    \cexpt{S}{A = a} & = m \cdot \frac{a - 1}{2m - 1} \cdot \frac{a}{2m}\\
                     & = \frac{a(a - 1)}{2(2m - 1)}
  \end{align*}
  Note that $\cexpt{S}{A = a}$ does not depend on $p$.
\end{solution}

\begin{example}
  A coin that has probability of heads equal to $p$ is tossed successively
  and independently until a head comes twice in a row or a tail comes
  twice in a row. Find the expected value of the number of tosses.
\end{example}

\begin{solution}
  Let $X$ be number of tosses until the game is over and let $H_k$
  (or $T_k$) be the event that a head (or a tail, respectively)
  comes at the $k$th toss. Since $H_1$ and $T_1$ form a partition of the
  sample space, by total expectation theorem, we have
  \begin{equation*}
    \expt{X} = p \cexpt{X}{H_1} + (1 - p) \cexpt{X}{T_1}
  \end{equation*}
  Using the total expectation theorem again, we have
  \begin{align*}
    \cexpt{X}{H_1} & = p \cexpt{X}{H_1 \cap H_2} +
                     (1 - p) \cexpt{X}{H_1 \cap T_2}\\
                   & = 2p + (1 - p) (1 + \cexpt{X}{T_1})
  \end{align*}
  Since $\cexpt{X}{H_1 \cap H_2} = 1$ as game ends. Also
  $\cexpt{X}{H_1 \cap T_2} = 1 + \cexpt{X}{T_1}$ since the last toss only
  matters and the game proceeds. Similarly we have
  \begin{equation*}
    \cexpt{X}{T_1} = 2(1 - p) + p (1 + \cexpt{X}{H_1})
  \end{equation*}
  Combining the above equations we have
  \begin{equation*}
    \cexpt{X}{H_1} = \frac{2 + {(1 - p)}^2}{1 - p(1 - p)}
  \end{equation*}
  And
  \begin{equation*}
    \cexpt{X}{T_1} = \frac{2 + p^2}{1 - p(1 - p)}
  \end{equation*}
  Thus
  \begin{equation*}
    \expt{X} = \frac{2 + p(1 - p)}{1 - p(1 - p)}
  \end{equation*}
\end{solution}
\end{document}
