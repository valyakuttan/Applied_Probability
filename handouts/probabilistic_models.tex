\documentclass{tufte-handout}
\usepackage{amsmath, amsfonts, amsthm}

%% environments
\theoremstyle{definition} \newtheorem{definition}{Definition}
\newtheorem{theorem}{Theorem}
\theoremstyle{remark} \newtheorem{remark}{Remark}

%% commands
\newcommand{\prob}[1]{\mathbf{P}\left(#1\right)}

\author{John N. Tsitsiklis}
\title{Probabilistic Models}
\begin{document}
\maketitle

\section{Sets}
\begin{definition}[De Morgan's laws]
  \begin{equation}
    \left( \bigcup_n S_n \right)^c = \bigcap_n {S_n}^c
  \end{equation}
  \begin{equation}
    \left( \bigcap_n S_n \right)^c = \bigcup_n {S_n}^c
  \end{equation}
\end{definition}

\begin{theorem}
  \begin{equation}
    A \cup \left( \cap_{n=1}^{\infty} B_n\right) =
    \cap_{n=1}^{\infty} \left( A \cup B_n \right)
  \end{equation}
  \begin{equation}
    A \cap \left( \cup_{n=1}^{\infty} B_n\right) =
    \cup_{n=1}^{\infty} \left( A \cap B_n \right)
  \end{equation}

\end{theorem}

\section{Probability Laws}
\begin{definition}[Probability Axioms]\ \\
  \begin{enumerate}
  \item \textbf{Nonnegativity} $\prob{A} \geq 0$, for every event
    $A$.
  \item \textbf{Additivity} If $A$ and $B$ are two disjoint events then
    \begin{equation*}
      \prob{A \cup B} = \prob{A} + \prob{B}
    \end{equation*}
  \item \textbf{Normalization} $\prob{\Omega} = 1$, where $\Omega$ is
    the \emph{sample space}.
  \end{enumerate}
\end{definition}

\begin{definition}[Discrete Probability Law]
  If the sample space consists of a finite number of possible outcomes,
  then the probability law is specified by the probabilities of the
  events that consists of a single element. In particular, the probability
  of any event $\{s_1, s_2, \ldots s_n\}$ is the sum of the probabilities
  of its elements:
  \begin{equation*}
    \prob{\{s_1, s_2, \ldots s_n\}} = \prob{s_1} + \prob{s_2} + \cdots +
    \prob{s_n}
  \end{equation*}
\end{definition}

\begin{definition}[Discrete Uniform Probability Law]
  If the sample space consists of $n$ possible outcomes which are equally
  likely, then the probability of any event $A$ is given by
  \begin{equation*}
    \prob{A} = \frac{\text{number of elements of }A}{n}
  \end{equation*}
\end{definition}

\section{Properties of Probability Laws}
Consider a probability law and let $A, B$ and $C$ be events:
\begin{enumerate}[(a)]
\item If $A \subset B$, then $\prob{A} \leq \prob{B}$
\item $\prob{A \cup B} = \prob{A} + \prob{B} - \prob{A \cap B}$
\item $\prob{A \cup B} \leq \prob{A} + \prob{B}$
\item $\prob{A \cup B \cup C} = \prob{A} + \prob{A^c \cap B} +
  \prob{A^c \cap B^c \cap C}$
\end{enumerate}

\begin{definition}
  A \emph{partition} of the sample space $\Omega$ is a collection of
  disjoint events $S_1, S_2, \ldots S_n$ such that $\Omega =
  \cup_{i=1}^n S_i$. Then
  \begin{equation*}
    \prob{A} = \sum_{i=1}^n \prob{A \cap S_i}
  \end{equation*}
\end{definition}

\begin{definition}[Bonferroni's inequality]
  We have
  \begin{enumerate}[(a)]
  \item for any two events $A$ and $B$
    \begin{equation*}
      \prob{A \cap B} \geq \prob{A} + \prob{B} - 1
    \end{equation*}
  \item for $n$ events $A_1, A_2, \ldots, A_n$
    \begin{equation*}
      \prob{A_1 \cap A_2 \cap \cdots \cap A_n} \geq \prob{A_1} +
      \prob{A_2} + \cdots + \prob{A_n} - (n - 1)
    \end{equation*}
  \end{enumerate}
\end{definition}

\begin{definition}[The inclusion-exclusion formula]
  We have
  \begin{enumerate}[(a)]
  \item for any two events $A$ and $B$
    \begin{equation*}
      \prob{A \cup B} = \prob{A} + \prob{B} - \prob{A \cup B}
    \end{equation*}
  \item for $n$ events $A_1, A_2, \ldots, A_n$. Let
    $S_1 = \{i | 1 \leq i \leq n\}$, $S_2 = \{(i_1, i_2) |
    1 \leq i_1 < i_2 \leq n\}$, and more generally, let $S_m$ be the set
    of all $m$-tuples $(i_1, i_2, \ldots, i_m)$ of indices that satisfy
    $1 \leq i_1 < i_2 < \cdots < i_m \leq n$. Then,
    \begin{align*}
      \prob{\cup_{k=1}^n A_k} & = \sum_{i \in S_1}\prob{A_i} \\
                              & - \sum_{(i_1, i_2) \in S_2}\prob{A_{i_1}
                                \cap A_{i_2}} \\
                              & + \sum_{(i_1, i_2, i_3) \in S_3}
                                \prob{A_{i_1} \cap A_{i_2} \cap A_{i_3}}
                                - \cdots \\
                              & + {(-1)}^{n - 1} \prob{\cap_{k=1}^n A_k}
      \end{align*}
    \end{enumerate}
\end{definition}

\section{Continuity property of probabilities}
\begin{theorem}
  Let $A_1, A_2, \ldots$ be an infinite sequence of events, which is
  \emph{monotonically increasing}, meaning $A_n \subset A_{n+1}$ for every
  $n$. Then
  \begin{equation*}
    \prob{\lim_{n \to \infty} \cup_{k=1}^n A_k} =
    \lim_{n \to \infty} \prob{A_n}
  \end{equation*}
\end{theorem}

\begin{proof}
  Let $B_1 = A_1$ and $B_n = A_n\cap A_{n-1}^c$ for $n \geq 2$. The events
  $B_n$ are disjoint, and we have $An = \cup_{k=1}^n B_k$. Let $A =
  \cup_{n=1}^\infty A_n$. Then $A = \cup_{k=1}^{\infty} B_k$. Now by the
  \emph{additivity axiom} we have
  \begin{align*}
    \prob{A} & = \sum_{k=1}^\infty \prob{B_k} \\
             & = \lim_{n \to \infty} \sum_{k=1}^n \prob{B_k} \\
             & = \lim_{n \to \infty} \prob{\cup_{k=1}^n B_k} \\
             & = \lim_{n \to \infty} \prob{A_n}
  \end{align*}
\end{proof}

\begin{theorem}
  Let $A_1, A_2, \ldots$ be an infinite sequence of events, which is
  \emph{monotonically decreasing}, meaning $A_{n+1} \subset A_n$ for every
  $n$. Then
  \begin{equation*}
    \prob{\lim_{n \to \infty} \cap_{k=1}^n A_k} =
    \lim_{n \to \infty} \prob{A_n}
  \end{equation*}
\end{theorem}

\end{document}
