\documentclass{tufte-handout}
\usepackage{amsmath, amsfonts, amsthm}

%% environments
\theoremstyle{definition} \newtheorem{definition}{Definition}
\newtheorem{theorem}{Theorem}
\theoremstyle{definition} \newtheorem{remark}{Remark}

%% commands
\newcommand{\prob}[1]{\mathbf{P}\left(#1\right)}
\newcommand{\cprob}[2]{\mathbf{P}\left(#1 \,|\, #2\right)}

\author{John N. Tsitsiklis}
\title{Probabilistic Models}
\begin{document}
\maketitle
\section{Conditional Probability}
\begin{definition}[Conditional Probability]
  The \emph{conditional probability} of an event $A$, given an event $B$
  with $\prob{B} > 0$, is defined by
  \begin{equation*}
    \prob{A | B} = \frac{\prob{A \cap B}}{\prob{B}},
  \end{equation*}
  and specifies a new probability law on the same sample space $\Omega$.
  In particular, all properties of probability laws remain valid for
  conditional probability laws.
\end{definition}

\begin{remark}
  If the possible outcomes are finitely many and equally likely, then
  \begin{equation*}
    \prob{A | B} = \frac{\text{number of elements of } A \cap B}
    {\text{number of elements of } B}
  \end{equation*}
\end{remark}

\begin{definition}[Multiplication Rule]
  Assuming that all the conditioning events have positive probability, we
  have
  \begin{equation*}
    \prob{\cap_{i=1}^n A_i} = \prob{A_1} \prob{A_2 | A_1}
    \prob{A_3 | A_1 \cap A_2} \cdots \prob{A_n | \cap_{i=1}^{n-1} A_i}
  \end{equation*}
\end{definition}

\section{Total probability theorem and Baye's rule}
\begin{definition}[Total Probability Theorem]
  Let $A_1, A_2, \ldots, A_n$ be disjoint events that form a partition of
  the sample space and $\prob{A_i} > 0$, for all $i$. Then, for any event
  $B$, we have
  \begin{align*}
    \prob{B} & = \prob{A_1 \cap B} + \cdots + \prob{A_1 \cap B} \\
             & = \prob{A_1} \prob{B | A_1} + \cdots + \prob{A_n}
               \prob{B | A_n}
  \end{align*}
\end{definition}

\begin{definition}[Baye's Rule]
  Let $A_1, A_2, \ldots, A_n$ be disjoint events that form a partition of
  the sample space, and assume that $\prob{A_i} > 0$, for all $i$. Then,
  for any event $B$ such that $\prob{B} > 0$, we have
  \begin{align*}
    \prob{A_i | B} & = \frac{\prob{A_i} \prob{B | A_i}}{\prob{B}} \\
                   & = \frac{\prob{A_i} \prob{B | A_i}}
                     {\prob{A_1} \prob{B | A_1} + \cdots + \prob{A_n}
                     \prob{B | A_n}}
  \end{align*}
\end{definition}

\section{Independence}
\begin{definition}[Independence]
  Two events $A$ and $B$ are said to be \emph{independent} if
  \begin{equation*}
    \prob{A \cap B} = \prob{A} \prob{B}
  \end{equation*}
\end{definition}

\begin{remark}
  If $A$, $B$ are independent, then $A$, $B^c$ and $A^c$, $B^c$ are
  also independent.
\end{remark}

\subsection{Conditional Independence}
\begin{definition}[Conditional Independence]
  Two events $A$ and $B$ are said to be \emph{conditionally independent}
  given another event $C$ with $\prob{C} > 0$, if
  \begin{equation*}
    \cprob{A \cap B}{C} = \cprob{A}{C} \cprob{B}{C}
  \end{equation*}
\end{definition}

\begin{remark}
  If $\prob{B \cap C} > 0$, conditional independence is equivalent to the
  condition
  \begin{equation*}
    \cprob{A}{B \cap C} = \cprob{A}{C}
  \end{equation*}
\end{remark}

\begin{remark}
  Independence does not imply conditional independence, and vice versa.
\end{remark}

\subsection{Independence of a Collection of Events}
\begin{definition}[Independence of Several Events]
  We say that the events $A_1, A_2, \ldots, A_n$ are \emph{independent} if
  \begin{equation*}
    \prob{\bigcap_{i \in S} A_i} = \prod_{i \in S} \prob{A_i} \quad
    \text{for every subset $S$ of } \{1, 2, \ldots, n\}.
  \end{equation*}
\end{definition}


\end{document}
