\documentclass{tufte-handout}
\usepackage{amsmath, amsfonts, amsthm}

%% environments
\theoremstyle{definition} \newtheorem{definition}{Definition}
\newtheorem{theorem}{Theorem}
\theoremstyle{definition} \newtheorem{remark}{Remark}
\theoremstyle{definition} \newtheorem{example}{Example}

%% commands
\newcommand{\prob}[1]{\mathbf{P}\left(#1\right)}
\newcommand{\cprob}[2]{\mathbf{P}\left(#1 \,|\, #2\right)}

\author{John N. Tsitsiklis}
\title{Counting}
\begin{document}
\maketitle
\section{Basics}

\subsection{The Counting Principle}
Consider a process that consists of $r$ stages. Suppose that:
\begin{enumerate}[(a)]
\item There are $n_1$ possible results at the first stage.
\item For every possible result at the first stage, there are $n_2$
  possible results at the second stage.

\item More generally, for an sequence of possible results at the first
  $i - 1$ stages, there are $n_i$ possible results at the $i^{\text{th}}$
  stage. Then, the total number of possible results of the $r$-stage
  process is
  \begin{equation*}
    n_1 n_2 \cdots n_r
  \end{equation*}
\end{enumerate}

\subsection{$k$-permutations}
We start with $n$ distinct objects, and let $k$ be some positive integer,
with $0 \leq k \leq n$. We wish to count the number of different ways
that we can pick $k$ out of these $n$ objects and arrange them in a
sequence.
\begin{equation*}
  k\text{-permutations} = \frac{n!}{(n-k)!}
\end{equation*}

\subsection{Combinations}
We start with $n$ distinct objects, and let $k$ be some positive integer,
with $0 \leq k \leq n$. We wish to count the number of $k$-element
subsets of a given $n$-element set. The number of possible combinations
is
\begin{equation*}
  \frac{n!}{k!(n-k)!} \text{ which is same as } \binom{n}{k}
\end{equation*}

\subsection{Partitions}
We are given an $n$-element set and nonnegative integers
$n_1, n_2, \ldots, n_r$, whose sum is equal to $n$. We consider partitions
of the set into $r$ disjoint subsets, with the $i^{\text{th}}$ subset
containing exactly $n_i$ elements. The number of partitions is
\begin{equation*}
  \frac{n!}{n_1! n_2! \cdots n_r!} \text{ which is denoted by }
  \binom{n}{n_1, n_2, \ldots, n_r}
\end{equation*}

\section{Examples}
\begin{example}
  Ninety students, including Joe and Jane, are to be split into three
  classes of equal size, and this is to be done at random. What is the
  probability that Joe and Jane end up in the same class?
\end{example}

 We place Joe in one class. Regarding Jane, there are $89$ possible
 slots, and only $29$ place her in the same class as Joe. Thus the
 answer is $\frac{29}{89}$.

 \begin{example}[Hypergeometric probabilities]
   An urn contains $n$ balls, out of which $m$ are red. We select $k$ of
   the balls at random, without replacement. What is the probability that
   $i$ of the selected balls are red?
 \end{example}

 The sample space consists of $\binom{n}{k}$ different ways of selecting
 $k$ out of $n$ balls. For the event of interest to occur, we have to
 select $i$ out of the $m$ red balls and also select $k - i$ balls out of
 the $n - m$ balls, which are not red. Thus the desired probability is
 \begin{equation*}
   \frac{\binom{m}{i} \binom{n - m}{k - i}}{\binom{n}{k}}
 \end{equation*}
 for $i \geq 0$ satisfying $i \leq m, i \leq k$ and $k - i \leq n - m$.
 For all other $i$, the probability is zero.
\end{document}
